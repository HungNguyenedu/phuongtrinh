\begin{Answer}{1}
    Phương trình đã cho tương đương: \\
    $ \begin{cases}
        2x -5 \geq 0 \\
        (-x^2 +4x -3) = (2x-5)^2
    \end{cases} $ \\
    $ \Leftrightarrow
        \begin{cases}
            x \geq \dfrac{5}{2} \\
            -x^2 + 4x -3 = 4x^2 -20x + 25
        \end{cases} $ \\
    $ \Leftrightarrow
    \begin{cases}
        x \geq \dfrac{5}{2 } \\
        5x^2 -24x +28 =0
    \end{cases} $ \\
    $ \Leftrightarrow
    \begin{cases}
        x \geq \dfrac{5 }{2 } \\
        \left[
            \begin{array}{l}
                x=2 \\
                x=\dfrac{14 }{5 }
            \end{array}
        \right.
    \end{cases} $ \\
    $ \Leftrightarrow  x=\dfrac{14 }{5 } $ \\
    Kết luận: Vậy phương trình có nghiệm duy nhất: $ x=\dfrac{14 }{5 }   $.
\end{Answer}
\begin{Answer}{2}
Phương trình đã cho tương đương: $ \sqrt{x^2 + x +2 } = 3 - x $ \\
$ \Leftrightarrow
\begin{cases}
    3-x \geq 0 \\
    x^2 + x + 2 = (3-x )^2
\end{cases} $ \\
$ \Leftrightarrow
\begin{cases}
    x \leq 3 \\
    x^2 + x + 2 = 9 -6x + x^2
\end{cases} $ \\
$ \Leftrightarrow
\begin{cases}
    x \leq 3 \\
    x=1
\end{cases} $
$ \Leftrightarrow x=1 $ \\
Kết luận: Vậy phương trình có nghiệm duy nhất: $ x=1 $.
\end{Answer}
\begin{Answer}{3}
    Phương trình đã cho tương đương:
    $
    \begin{cases}
        3-2x -x^2 \geq 0 \\
        7-x^2 + x \sqrt{x+5} = 3-2x -x^2
    \end{cases} $ \\

   $ \Leftrightarrow
    \begin{cases}
        -3 \leq x \leq 1 \\
        x \sqrt{x+5} = -4 -2x
    \end{cases} $  \\

     $ \Leftrightarrow
    \begin{cases}
        -3 \leq x \leq 1 \\
         \sqrt{x+5} = \dfrac{-4-2x}{x}
    \end{cases} $ \\

     $ \Leftrightarrow
    \begin{cases}
        -3 \leq x \leq 1 \\
        \dfrac{-4-2x}{x} \geq 0 \\
        x^2( x+5) = (-4-2x)^2
    \end{cases} $  \\

     $ \Leftrightarrow
    \begin{cases}
        -3 \leq x \leq 1 \\
        -2 \leq x <0 \\
        x^3 +5x^2 = 16 +16x +4x^2
    \end{cases} $ \\

     $ \Leftrightarrow
    \begin{cases}
        -2 \leq x <0 \\
        x^3 +x^2 -16x -16  =   0
    \end{cases} $ \\

    $ \Leftrightarrow
    \begin{cases}
        -2 \leq x <0 \\
        \left[
            \begin{array}{l}
                x=-1 \\
                x = \pm 4
            \end{array}
        \right.
    \end{cases} $ \\
    $ \Leftrightarrow x=-1 $ \\
    Kết luận: Vậy phương trình có nghiệm duy nhất: $ x=-1 $.
\end{Answer}
\begin{Answer}{4}
    Điều kiện: $ \begin{cases}
        3x -2 \geq 0 \\
        x + 7 \geq 0
    \end{cases} $
    $
    \Leftrightarrow
    \begin{cases}
        x \geq \dfrac{2}{3 } \\
        x \geq -7
    \end{cases} $
    $
    \Leftrightarrow
        x \geq \dfrac{2}{3 } $ \\

    Với điều kiện trên phương trinh tương đương với:
    $ \sqrt{3x -2 } = 1 + \sqrt{x+7}  $ \\
    $ \Leftrightarrow 3x -2 = x+8 + 2 \sqrt{x+7 }  $ \\
    $ \Leftrightarrow x -5 = \sqrt{x+7} $ \\
    $ \Leftrightarrow \begin{cases}
        x \geq 5 \\
        x^2 -10x + 25 = x +7
    \end{cases} $ \\
    $ \Leftrightarrow \begin{cases}
        x \geq 5 \\
        x^2 -11x + 18 =0
    \end{cases} $ \\
    $ \Leftrightarrow \begin{cases}
        x \geq 5 \\
        \left[
            \begin{array}{l}
                x=9 \\
                x=2
            \end{array}
        \right.
    \end{cases} $ \\
    $ \Leftrightarrow x=9 $ \\
    Kết hợp điều kiện ta được $ x=9 $ là nghiệm duy nhất của phương trình.
\end{Answer}
\begin{Answer}{5}
Điều kiện: $ x \geq 0 $.
Với điều kiện trên phương trình tương đương: $ \sqrt{x+8 } = \sqrt{x } + \sqrt{x + 3}  $ \\
$ \Leftrightarrow x+8 = 2x +3 + 2 \sqrt{x(x+3 ) } $ \\
$ \Leftrightarrow 5-x = 2 \sqrt{x^2 + 3x } $ \\
$ \Leftrightarrow \begin{cases}
    x \leq 5 \\
    25 -10x + x^2 = 4x^2 + 12x
\end{cases} $ \\
$ \Leftrightarrow \begin{cases}
    x \leq 5 \\
    3x^2 +22x -25 =0
\end{cases} $ \\
$ \Leftrightarrow \begin{cases}
    x \leq 5 \\
    \left[
        \begin{array}{l}
            x =1 \\
            x= -\dfrac{25 }{3 }
        \end{array}
    \right.
\end{cases} $ \\
$ \Leftrightarrow \left[
    \begin{array}{l}
        x =1 \\
        x=-\dfrac{25 }{3 }
    \end{array}
\right. $ \\
Kết hợp điều kiện ta được $x=1$ là nghiệm duy nhất của phương trình.
\end{Answer}
\begin{Answer}{6}

\end{Answer}
\begin{Answer}{7}

\end{Answer}
\begin{Answer}{8}

\end{Answer}
\begin{Answer}{9}

\end{Answer}
\begin{Answer}{10}

\end{Answer}
