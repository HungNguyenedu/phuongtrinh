%%%%%%%%%%%%%%%%%%%%%%%%%%%%%%%%%%%%%%%%%%%%%%%%%%%%%%%%%%%%%%%%%%%%%%%%%%%%%%%%
%			                 vi du				       %
%%%%%%%%%%%%%%%%%%%%%%%%%%%%%%%%%%%%%%%%%%%%%%%%%%%%%%%%%%%%%%%%%%%%%%%%%%%%%%%%
\begin{vd}
    Giải phương trình: $ \sqrt[3]{x-1}+\sqrt[3]{x-2} = \sqrt[3]{2x-3}$
\end{vd}
\begin{center}
    Giải:
\end{center}

Phương trình đã cho tương đương: \\
$ \left( \sqrt[3]{x-1}+\sqrt[3]{x-2} \right)^3 = \left( \sqrt[3]{2x-3 } \right)^3 $\\ 
$ \Leftrightarrow x -1 + x -2 + 3 \sqrt[3]{x-1}\sqrt[3]{x-2}\left(  \sqrt[3]{x-1}+\sqrt[3]{x-2}\right) = 2x -3 $ \\
$ \Rightarrow   \sqrt[3]{x-1}\sqrt[3]{x-2}  \sqrt[3]{2x-3} =0 $ \\
$ \Leftrightarrow \left[
    \begin{array}{l}
        x-1 =0 \\
        x-2 =0 \\
        2x -3 =0
    \end{array}
\right. $ \\
$ \Leftrightarrow \left[
    \begin{array}{l}
        x=1 \\ x=2 \\ x= \dfrac{3}{2}
    \end{array}
\right. $ \\
Thử lại ta thấy nghiệm của phương trình là: $ x=1; x=2 ; x=\dfrac{3}{2}$.


