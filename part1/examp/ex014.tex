%%%%%%%%%%%%%%%%%%%%%%%%%%%%%%%%%%%%%%%%%%%%%%%%%%%%%%%%%%%%%%%%%%%%%%%%%%%%%%%%
%			                 vi du				       %
%%%%%%%%%%%%%%%%%%%%%%%%%%%%%%%%%%%%%%%%%%%%%%%%%%%%%%%%%%%%%%%%%%%%%%%%%%%%%%%%
\begin{vd}
  Giải phương trình: $ 3\left( 2+\sqrt{x-2} \right) =2x +\sqrt{x+6} $
\end{vd}
\begin{center}
    Giải:
\end{center}

Điều kiện: $ x \geq 0 $ \\
Với điều kiện trên phương trình tương đương : \\
$ 3 \sqrt{x-2} -\sqrt{x+6} = 2(x-3) $\\
$ \Leftrightarrow 9 (x-2) - (x+6) = 2(x-3) \left( 3 \sqrt{x-2 } + \sqrt{x+6 } \right) $ \\
$ \Leftrightarrow 8(x - 3) =2(x-3) \left( 3\sqrt{x-2 } + \sqrt{x+6 } \right) $ \\
$ \Leftrightarrow (x-3) ( 4 - \left( 3\sqrt{x-2 } + \sqrt{x+6} \right) $ \\
$ \left[
    \begin{array}{lr}
        x  = 3 & \\
        3\sqrt{x-2 } + \sqrt{ x+ 6} -4 =0 & (1)  
    \end{array}
\right. $\\
Ta có $(1) \Leftrightarrow 3\sqrt{x-2 } + \sqrt{x+6} =4 $\\
$ \Leftrightarrow 9(x-2) + x+6 + 6\sqrt{(x-2)(x+6)} =16 $\\
$ \Leftrightarrow 10x - 12 +6 \sqrt{x^2 +4x -12 } =16 $ \\
$ \Leftrightarrow 3 \sqrt{x^2 +4x -12 } = 14 -5x $\\
$ \Leftrightarrow \begin{cases}
    x \leq  \dfrac{14}{5} \\
    9(x^2 + 4x -12) = (14 -5x)^2 
\end{cases} $ \\
$ \Leftrightarrow \begin{cases}
    x \leq \dfrac{14}{5} \\
    16x^2 - 176x +304 = 0
\end{cases} $ \\
$ \Leftrightarrow \begin{cases}
     x \leq \dfrac{14}{5} \\
     x^2 -11x + 19 =0
\end{cases} $ \\

$ \Leftrightarrow \begin{cases}
     x \leq \dfrac{14}{5} \\
     \left[
         \begin{array}{l}
             x = \dfrac{11+3 \sqrt{5}}{2} \\
             x = \dfrac{11-3 \sqrt{5}}{2} 
         \end{array}
     \right. 
     
\end{cases} $ \\ 
$ \Leftrightarrow 
             x = \dfrac{11-3 \sqrt{5}}{2}  $ \\ 

Kết luận: Vậy phương trình có 2 nghiệm là: $ x=3 ;  x = \dfrac{11-3 \sqrt{5}}{2}   $.

