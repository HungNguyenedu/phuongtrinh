%%%%%%%%%%%%%%%%%%%%%%%%%%%%%%%%%%%%%%%%%%%%%%%%%%%%%%%%%%%%%%%%%%%%%%%%%%%%%%%%
%			                 vi du				       %
%%%%%%%%%%%%%%%%%%%%%%%%%%%%%%%%%%%%%%%%%%%%%%%%%%%%%%%%%%%%%%%%%%%%%%%%%%%%%%%%
\begin{vd}
  Giải phương trình: $ \sqrt{2x^2 +x +6} + \sqrt{x^2 +x +2} = x + \frac{4}{x } $
\end{vd}
\begin{center}
    Giải:
\end{center}
$\sqrt{2x^2 +x +6} + \sqrt{x^2 +x +2} = x + \frac{4}{x } \text{ (1) }$
Điều kiện: $ x \neq 0  $ \\
Để $x$ là nghiệm của phương trình thì $x >0$. \\
Phương trinh đã cho tương đương:\\
$     \dfrac{x^2 +4}{\sqrt{2x^2 +x +6} - \sqrt{x^2 +x +2}} =     \dfrac{x^2 +4}{x}  $ \\
$ \Leftrightarrow \sqrt{2x^2 +x +6} - \sqrt{x^2 +x +2} = x \text{ (2) }$ \\
Kết hợp giữa (1) và (2) ta được phương trình: $ 2     \sqrt{x^2+x +2 } =     \dfrac{4}{x } $ \\
$ \Leftrightarrow x \sqrt{x^2 +x +2 } = 2 $ \\
$  \Leftrightarrow x^4 + x^3 + 2x^2 -4 =0 $ \\
$ \Leftrightarrow (x-1)( x^3 +2x^2 +4x +4) =0 $ \\
$ \Leftrightarrow x = 1 $ ( do $ x^3 +2x^2 +4x +4 > 0$  $  \forall x >0 $ ) \\
Kết luận: Vậy phương trình có nghiệm duy nhất: $ x=1 $.
