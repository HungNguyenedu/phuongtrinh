%%%%%%%%%%%%%%%%%%%%%%%%%%%%%%%%%%%%%%%%%%%%%%%%%%%%%%%%%%%%%%%%%%%%%%%%%%%%%%%%
%			                 Bai tap				       %
%%%%%%%%%%%%%%%%%%%%%%%%%%%%%%%%%%%%%%%%%%%%%%%%%%%%%%%%%%%%%%%%%%%%%%%%%%%%%%%%
\begin{baitap}
 Giải phương trình: $ \sqrt{x(x-1)} + \sqrt{x(x+2)} = 2x   $
\begin{loigiai1}
Điều kiện: $ \begin{cases}
    x(x-1 ) \geq 0 \\
    x(x+2 ) \geq 0 \\
    x \geq 0
\end{cases} 
\Leftrightarrow \begin{cases}
    x \leq 0 \vee 1 \leq x \\
    x \leq -2  \vee 0 \leq x \\
    x \geq 0
\end{cases} 
\Leftrightarrow  \left[
    \begin{array}{l}
        x =0 \\
        x \geq 1
    \end{array}
\right. $ \\
Ta có $ x=0 $ là nghiệm của phương trình đã cho.\\
Với $ x \geq 1 $ thì phương trình đã cho tương đương với : \\
$ \sqrt{x } \sqrt{x-1 } + \sqrt{x } \sqrt{x+2 } = 2 \sqrt{x^2}     $ \\
$ \Leftrightarrow \sqrt{x-1 } + \sqrt{x +2 } = 2 \sqrt{x}    $ \\
$ \Leftrightarrow 2x + 1 + 2 \sqrt{(x-1)(x+2)} = 4x $ \\
$ \Leftrightarrow \sqrt{x^2 +x -2 } = x - \dfrac{1 }{2 }  $ \\
$ \Leftrightarrow \begin{cases}
    x \geq \dfrac{1}{2 } \\
    x^2 +x -2 = x^2 -x + \dfrac{1 }{4 }  
\end{cases} $ \\
$ \Leftrightarrow 
\begin{cases}
    x \geq \dfrac{1 }{2}   \\
    x = \dfrac{9 }{8}  
\end{cases} $ \\
$ \Leftrightarrow x =\dfrac{9}{8}   $ \\

Kết luận: Vậy phương trình có 2 nghiệm là: $ x=0 ; x=\dfrac{9}{8}    $.

\end{loigiai1}

\end{baitap}
