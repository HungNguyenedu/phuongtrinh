%%%%%%%%%%%%%%%%%%%%%%%%%%%%%%%%%%%%%%%%%%%%%%%%%%%%%%%%%%%%%%%%%%%%%%%%%%%%%%%%
%			                 Bai tap				       %
%%%%%%%%%%%%%%%%%%%%%%%%%%%%%%%%%%%%%%%%%%%%%%%%%%%%%%%%%%%%%%%%%%%%%%%%%%%%%%%%
\begin{baitap}
 Giải phương trình: $ \sqrt{x+3 } + \sqrt{3x +1 } = \sqrt{2x+2 } + 2 \sqrt{ x}     $
\begin{loigiai1}
Điều kiện: $ \begin{cases}
    x +3 \geq 0 \\
    3x +1 \geq 0 \\
    2x +2 \geq 0 \\
    x \geq 0
\end{cases} 
\Leftrightarrow 
    \begin{cases}
        x \geq -3 \\
        x \geq - \frac{1}{3} \\
        x \geq -1 \\
        x \geq 0
    \end{cases} \Leftrightarrow x \geq 0 $ \\
    Với điều kiện trên phương trình tương đương: \\
    $ \sqrt{x+3 } - 2 \sqrt{x } = \sqrt{2x +2 } - \sqrt{3x + 1} $ \\
    $ \Leftrightarrow 5x +3 -4 \sqrt{x^2 + 3x } = 5x + 3 - 2 \sqrt{(2x+2)(3x+1)}$ \\
    $ \Leftrightarrow 2 \sqrt{x^2 + 3x } = \sqrt{6x^2 + 8x + 2 } $ \\
    $ \Leftrightarrow 4x^2 + 12x = 6x^2 + 8x + 2 $ \\
    $ \Leftrightarrow x^2 -2x +1 =0 $ \\
    $ \Leftrightarrow x = 1 $ \\
    Thử lại ta thấy $x=1$ là nghiệm duy nhất của phương trình.
\end{loigiai1}

\end{baitap}
