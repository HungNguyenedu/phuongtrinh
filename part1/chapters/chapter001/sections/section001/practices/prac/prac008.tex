%%%%%%%%%%%%%%%%%%%%%%%%%%%%%%%%%%%%%%%%%%%%%%%%%%%%%%%%%%%%%%%%%%%%%%%%%%%%%%%%
%			                 Bai tap				       %
%%%%%%%%%%%%%%%%%%%%%%%%%%%%%%%%%%%%%%%%%%%%%%%%%%%%%%%%%%%%%%%%%%%%%%%%%%%%%%%%
\begin{baitap}
 Giải phương trình: $\sqrt{4x +5 } + \sqrt{3x +1 } = \sqrt{2x +7 } + \sqrt{x+3 } $
\begin{loigiai1}
    Điều kiện: $ \begin{cases}
        x \geq -\dfrac{5 }{4 } \\
        x \geq - \dfrac{1 }{3 } \\
        x \geq - \dfrac{7}{2}  \\
        x \geq -3
    \end{cases} \Leftrightarrow x \geq -\dfrac{1 }{3 } $ \\
    Với điều kiện trên phương trình tương đương với: \\
    $\sqrt{4x +5 } -   \sqrt{x+3 }  =  \sqrt{2x +7 }- \sqrt{3x +1 }$ \\
$ \Leftrightarrow 5x +8 -2 \sqrt{(4x+5 )(x+3 ) } = 5x +8 -2 \sqrt{(2x+7)(3x+1) } $ \\
$ \Leftrightarrow \sqrt{4x^2 + 17x + 15 } = \sqrt{6x^2 + 23x + 7}  $ \\
$ \Leftrightarrow 4x^2 + 17x + 15 = 6x^2 +23 x +7 $ \\
$ \Leftrightarrow 2x^2 +6x -8 =0 $ \\
$ \Leftrightarrow x=1 \vee x= -4 $ \\
$ \Leftrightarrow x=1 $ \\
Thử lại ta thấy $ x=1 $ là nghiệm duy nhất của phương trình.
\end{loigiai1}

\end{baitap}
