\part{PHƯƠNG TRÌNH VÔ TỈ}%
\minitoc %

\thispagestyle{empty}
\Opensolutionfile{loigiaichung}[baitapC1]
\vspace*{1cm}
\chapter{Phương pháp giải phương trình vô tỉ}
\section{Phương trình vô tỉ giải được bằng phương pháp tương đương .}
  \subsection{Phương pháp giải }
    Chuyển vế đổi dấu để hai vế không âm, sau đó bình phương hai vế (ta được phương trình tương đương) để khử căn thức, đưa về phương trình đại số, trong đó: 
        \begin{itemize}
          \item Phương trình có dạng $	\sqrt{A}=B \Leftrightarrow \begin{cases} B \geq 0 \\ A = B^2 \end{cases}$
          \item Ta có thể bình phương mà không cần quan tâm tới điều kiện hai vế phải tương đương (ta được phương trình hệ quả) để khử căn thức, tuy nhiên sau khi giải ra nghiệm ta phải thử lại nghiệm.

        
        \end{itemize} 

      \subsection{Ví dụ minh họa}
      %%%%%%%%%%%%%%%%%%%%%%%%%%%%%%%%%%%%%%%%%%%%%%%%%%%%%%%%%%%%%%%%%%%%%%%%%%%%%%%%%%%%%%%
      %                         vidu                                                        %
      %%%%%%%%%%%%%%%%%%%%%%%%%%%%%%%%%%%%%%%%%%%%%%%%%%%%%%%%%%%%%%%%%%%%%%%%%%%%%%%%%%%%%%%
      	      %%%%%%%%%%%%%%%%%%%%%%%%%%%%%%%%%%%%%%%%%%%%%%%%%%%%%%%%%%%%%%%%%%%%%%%%%%%%%%%%%%%%%%%
      %                         vidu                                                        %
      %%%%%%%%%%%%%%%%%%%%%%%%%%%%%%%%%%%%%%%%%%%%%%%%%%%%%%%%%%%%%%%%%%%%%%%%%%%%%%%%%%%%%%%
      
          \begin{vd}
            Giải phương trình: $ \sqrt{2x-3} = x-3 $

          \end{vd}
          \begin{center}

            Giải:
         \end{center}
          Phương trình đã cho tương đương với: \\
         $
            \begin{cases}
                x - 3 \geq 0 \\
                2x- 3 = (x-3)^2
            \end{cases}
          $ 
        $    \Leftrightarrow $
          $
            \begin{cases}
                x \geq 3 \\
                2x - 3 = x^2 -6x +9
            \end{cases}
          $ \\
        $    \Leftrightarrow $
         $
            \begin{cases}
                x  \geq 3 \\
                x^2 - 8x + 12 =0
            \end{cases}
          $ 
        $    \Leftrightarrow $
         $
            \begin{cases}
                x  \geq 3 \\
                \left[ \begin{array} {l}
                    x=6 \\ 
                    x=2
                  \end{array}
                \right.
                
            \end{cases}
          $ 
        $    \Leftrightarrow $
        $
            x=6.
        $
         \\        
         Kết luận: vậy phương trình có một nghiệm là $x=6$

      	      %%%%%%%%%%%%%%%%%%%%%%%%%%%%%%%%%%%%%%%%%%%%%%%%%%%%%%%%%%%%%%%%%%%%%%%%%%%%%%%%%%%%%%%
      %                         vidu                                                        %
      %%%%%%%%%%%%%%%%%%%%%%%%%%%%%%%%%%%%%%%%%%%%%%%%%%%%%%%%%%%%%%%%%%%%%%%%%%%%%%%%%%%%%%%
          \begin{vd}
            Giải phương trình: $ x - \sqrt{2x -5} =4 $

          \end{vd}
          \begin{center}

            Giải:
         \end{center}
         
         Phương trình đã cho tương đương với : $ x - 4 = \sqrt{2x-5} $ \\
         $ \Leftrightarrow $
         $
            \begin{cases}
                x - 4 \geq 0 \\
                (x-4)^2 = 2x -5
            \end{cases}
         $
         $ \Leftrightarrow $
         $
            \begin{cases}
                x  \geq 4 \\
                x^2 -8x + 16 = 2x -5
            \end{cases}
         $
         $ \Leftrightarrow $
         $
            \begin{cases}
                x  \geq 4 \\
                x^2 -10x + 21 = 0
              \end{cases}
         $

         $ \Leftrightarrow $
         $
            \begin{cases}
                x  \geq 4 \\
                \left[
                  \begin{array} {l}
                     x = 7 \\ x=3
                  \end{array}
                \right.
              \end{cases}
         $
         $ \Leftrightarrow $
         $
              x =7.
         $
\\         Kết luận: Phương trình có một nghiệm là $ x=7 $.

      	
      %%%%%%%%%%%%%%%%%%%%%%%%%%%%%%%%%%%%%%%%%%%%%%%%%%%%%%%%%%%%%%%%%%%%%%%%%%%%%%%%%%%%%%%
      %                         vidu                                                        %
      %%%%%%%%%%%%%%%%%%%%%%%%%%%%%%%%%%%%%%%%%%%%%%%%%%%%%%%%%%%%%%%%%%%%%%%%%%%%%%%%%%%%%%%
      
          \begin{vd}
            Giải phương trình: $ \sqrt{-x^2 + 4x } + 2 = 2x$

          \end{vd}
          \begin{center}

            Giải:
         \end{center}
         
         Phương trình đã cho tương đương với : $ \sqrt{-x^2 +4x} = 2x -2 $ \\
         $ \Leftrightarrow $
         $
         \begin{cases}
          2x -2 \geq 0 \\ 
          -x^2 +4x = (2x-2)^2
         
         \end{cases}
         $
         $ \Leftrightarrow $
         $
         \begin{cases}
          x  \geq 1 \\ 
          -x^2 +4x = 4x^2 -8x + 4         
         \end{cases}
         $
         $ \Leftrightarrow $
         $
         \begin{cases}
          x  \geq 1 \\ 
          5x^2 -12x + 4 =0     
         \end{cases}
         $
         $ \Leftrightarrow $
         $
         \begin{cases}
          x  \geq 1 \\ 
          \left[ 
            \begin{array}{l}
                x=2 \\
                x=\dfrac{2}{5}
            
            \end{array}
          \right.
         \end{cases}
         $
         $ \Leftrightarrow $
         $ x=2 $. \\
         Kết luận: Phương trình có một nghiệm : $ x=2 $. 
         

      	
      %%%%%%%%%%%%%%%%%%%%%%%%%%%%%%%%%%%%%%%%%%%%%%%%%%%%%%%%%%%%%%%%%%%%%%%%%%%%%%%%%%%%%%%
      %                         vidu                                                        %
      %%%%%%%%%%%%%%%%%%%%%%%%%%%%%%%%%%%%%%%%%%%%%%%%%%%%%%%%%%%%%%%%%%%%%%%%%%%%%%%%%%%%%%%
      
          \begin{vd}
              Giải phương trình: $\sqrt{x+4} - \sqrt{1-x} = \sqrt{1-2x}$
          \end{vd}
          \begin{center}

            Giải:
         \end{center}
          Điều kiện:
          $ \begin{cases}
                \sqrt{x+4} \geq 0 \\
                \sqrt{1-x} \geq 0\\
                \sqrt{x+4} \geq \sqrt{1-x} \\
                \sqrt{1-2x} \geq 0     
             \end{cases} 
          $
         $ \Leftrightarrow 
                \begin{cases}
                    x \geq -4 \\
                    x \leq 1 \\
                    x \geq -\dfrac{3}{2} \\
                    x \leq \dfrac{1}{2}
                \end{cases}
         $
         $\Leftrightarrow
          -\dfrac{3}{2} \leq x \leq \dfrac{1}{2}$ \\
         Với điều kiện trên phương trình tương đương: \\
         $\sqrt{x+4} = \sqrt{1-2x} + \sqrt{1-x}$ \\
         $\Leftrightarrow
         x+4 = 1-2x + 1-x + 2 \sqrt{(1-2x)(1-x)}$ \\
         $\Leftrightarrow
         2x+1= \sqrt{1-3x+2x^2}$ \\
         $\Leftrightarrow
            \begin{cases}
                2x + 1 \geq 0 \\
                1 -3x +2x^2 = (2x+1)^2
            \end{cases}
         $
         $\Leftrightarrow
            \begin{cases}
                x \geq -\dfrac{1}{2} \\
                2x^2 + 7x =0 
            \end{cases}
         $
         $\Leftrightarrow x=0 $ \\
         So sánh với điêu kiện ta được nghiệm của phương trình là: $x=0$.

      	
         
      %%%%%%%%%%%%%%%%%%%%%%%%%%%%%%%%%%%%%%%%%%%%%%%%%%%%%%%%%%%%%%%%%%%%%%%%%%%%%%%%%%%%%%%
      %                         vidu                                                        %
      %%%%%%%%%%%%%%%%%%%%%%%%%%%%%%%%%%%%%%%%%%%%%%%%%%%%%%%%%%%%%%%%%%%%%%%%%%%%%%%%%%%%%%%
      
          \begin{vd}
              Giải phương trình: $\sqrt{3x+4} - \sqrt{2x+1} = \sqrt{x+3}$
          \end{vd}
          \begin{center}

            Giải:
         \end{center}
         Điều kiện: $ \begin{cases}
                            3x + 4 \geq 0 \\
                            2x + 1 \geq 0 \\
                            \sqrt{3x+4} \geq \sqrt{2x+1}\\
                            x+3 \geq 0
                      \end{cases}
                    $
                    $\Leftrightarrow
                    \begin{cases}
                        x \geq - \dfrac{4}{3} \\
                        x \geq -\dfrac{1}{2} \\
                        3x +4 \geq 2x +1 \\
                        x \geq 3
                    \end{cases}
                    $
                    $\Leftrightarrow
                    \begin{cases}
                        x \geq -\frac{4}{3} \\
                        x \geq -\frac{1}{2} \\
                        x \geq -3 \\
                        x \geq -3
                    \end{cases}
                    $
                    $\Leftrightarrow x \geq -\frac{1}{2}$ \\
                    Với điều kiện trên phương trình tương đương \\
                    $
                    \sqrt{3x+4} = \sqrt{x+3} + \sqrt{2x+1}
                    $ \\
                    $\Leftrightarrow
                    3x+4 = 3x +4 + 2\sqrt{(x+3)(2x+1)}
                    $ \\
                    $\Leftrightarrow 
                    \left[
                        \begin{array}{l}
                            x=-3 \\
                            x=-\frac{1}{2}
                        \end{array}
                    \right.
                    $ \\
                    So sánh với điều kiện ta được nghiệm phương trình là $x=-\frac{1}{2}$.


      	
         
      %%%%%%%%%%%%%%%%%%%%%%%%%%%%%%%%%%%%%%%%%%%%%%%%%%%%%%%%%%%%%%%%%%%%%%%%%%%%%%%%%%%%%%%
      %                         vidu                                                        %
      %%%%%%%%%%%%%%%%%%%%%%%%%%%%%%%%%%%%%%%%%%%%%%%%%%%%%%%%%%%%%%%%%%%%%%%%%%%%%%%%%%%%%%%
      
          \begin{vd}
              Giải phương trình: $\sqrt{3x+8} - \sqrt{3x+5} = \sqrt{5x-4} - \sqrt{5x-7}$
          \end{vd}
          \begin{center}

            Giải:
         \end{center}

		 Điều kiện: 
         $
         \begin{cases}
           3x + 8 \geq 0 \\
           3x + 5 \geq 0 \\
           3x + 8 \geq 3x + 5 \\
           5x -4 \geq 0 \\
           5x -7 \geq 0 \\
           5x -4 \geq 5x -7
         \end{cases}
         $
         $ \Leftrightarrow 
            \begin{cases}
              x \geq - \dfrac{8}{3} \\
              x \geq - \dfrac{5}{3} \\
              x \geq \dfrac{4}{5} \\
              x \geq \dfrac{7}{5}
            \end{cases}
        $
        $ \Leftrightarrow x \geq \dfrac{7}{5} $ \\
        Với điều kiện trên phương trình tương đương: \\
        $\sqrt{3x+8} +  \sqrt{5x-7}  = \sqrt{5x-4} + \sqrt{3x+5}$ \\
    $\Leftrightarrow  \sqrt{(3x+8)(5x-7)} = \sqrt{(5x-4)(3x+5)}$ \\
    $ \Leftrightarrow 15x^2 +19x -56 = 15x^2 +13x -20 $ \\
    $ \Leftrightarrow 6x = 36 \\ \Leftrightarrow x=6$ \\
    So sánh điều kiện ta được nghiệm của phương trình là: $x=6$.

		\begin{vd}
  Giải phương trình: $ x^2 + \sqrt{x+1} =1 $
\end{vd}
\begin{center}
    Giải:
\end{center}
Phương trình đã cho tương đương 
$ 1 - x^2  = \sqrt{x+1} $ \\

$ \Leftrightarrow 
    \begin{cases}
        x^2 \leq 1 \\
        (1 - x^2)^2 = x + 1
    \end{cases}
$ \\
$ \Leftrightarrow 
    \begin{cases}
      -1 \leq x \leq 1 \\
      x^4 - 2x^2 + 1 = x + 1
    \end{cases}
$ \\
$ \Leftrightarrow 
    \begin{cases}
      -1 \leq x \leq 1 \\
      x^4 -2x^2 -x =0
    \end{cases}
$ \\
$ \Leftrightarrow 
    \begin{cases}
      -1 \leq x \leq 1 \\
      x(x^3 -2x -1) = 0
    \end{cases}
$ \\
$ \Leftrightarrow 
    \begin{cases}
      -1 \leq x \leq 1 \\
      x(x+1)(x^2 -x -1) = 0
    \end{cases}
$ \\
$ \Leftrightarrow 
    \begin{cases}
      -1 \leq x \leq 1 \\
    \left[ 
        \begin{array}{l}
            x=0 \\
            x=1 \\
            x= \dfrac{1-\sqrt{5}}{2}
        \end{array}
    \right.
    \end{cases}
$\\
$ \Leftrightarrow 
    \left[ 
        \begin{array}{l}
            x=0 \\
            x=1 \\
            x= \dfrac{1-\sqrt{5}}{2}
        \end{array}
    \right.
$

Kết luận vậy phương trình có ba nghiệm là: $ x=0, x= -1, x=\dfrac{1-\sqrt{5}}{2}$

		%%%%%%%%%%%%%%%%%%%%%%%%%%%%%%%%%%%%%%%%%%%%%%%%%%%%%%%%%%%%%%%%%%%%%%%%%%%%%%%%
%			                 vi du				       %
%%%%%%%%%%%%%%%%%%%%%%%%%%%%%%%%%%%%%%%%%%%%%%%%%%%%%%%%%%%%%%%%%%%%%%%%%%%%%%%%
\begin{vd}
  Giải phương trình: $  \sqrt{x+3} + \sqrt{3x+1} = 2\sqrt{x} + \sqrt{2x+2}  $
\end{vd}
\begin{center}
    Giải:
\end{center}

Điều kiện: $ x \geq 0 $ \\
Với điều kiện trên phương trình tương đương với: \\
$ \sqrt{3x+1} - \sqrt{2x+2} = 2\sqrt{x} - \sqrt{x+3} $ \\
$ \Leftrightarrow 
    5x + 3 -2\sqrt{(3x+1)(2x+2)} = 5x + 3 -4\sqrt{x(x+3)}
$ \\
$ \Leftrightarrow 
    6x^2 + 8x +2 = 4x^2 + 12x 
$ \\
$ \Leftrightarrow 
    2x^2 - 4x + 2=0 
$ \\
$ \Leftrightarrow 
    x= 1 $ \\
    Thử lại thấy nghiệm $ x= 1 $ thỏa mãn. \\
    Kết luận: Vậy phương trình có nghiệm duy nhất: $ x=1 $.

		%%%%%%%%%%%%%%%%%%%%%%%%%%%%%%%%%%%%%%%%%%%%%%%%%%%%%%%%%%%%%%%%%%%%%%%%%%%%%%%%
%			                 vi du				       %
%%%%%%%%%%%%%%%%%%%%%%%%%%%%%%%%%%%%%%%%%%%%%%%%%%%%%%%%%%%%%%%%%%%%%%%%%%%%%%%%
\begin{vd}
  Giải phương trình: $ \dfrac{x^2}{3x-2} - \sqrt{3x-2} = 1-x $
\end{vd}
\begin{center}
    Giải:
\end{center}

Điều kiện: $ x > \dfrac{2}{3} $ \\
Với điều kiện trên phương trình tương đương với: \\
$ x^2 -3x +2 = (1-x)\sqrt{3x-2} $
$ \Leftrightarrow 
 (x-1)(x-2) = (1-x)\sqrt{3x-2}
$ \\
$ \Leftrightarrow 
    (x-1)(x-2 + \sqrt{3x-2}) =0
$ \\
$ \Leftrightarrow 
    \left[ 
        \begin{array}{lc} 
            x=1 \\
            x-2 + \sqrt{3x-2} =0 &(1)
        \end{array} 
    \right.
$ \\
Ta có phương trình $(1) \Leftrightarrow 
    2-x =\sqrt{3x-2}
$\\
$ \Leftrightarrow 
\begin{cases}
  	2 - x \geq 0 \\
    x^2 -4x +4 =3x -2
\end{cases} 
$ \\
$ \Leftrightarrow 
\begin{cases}
  	x \leq 2 \\
    x^2 -7x +6 =0
\end{cases}
$ \\
$ \Leftrightarrow 
\begin{cases}
  	x \leq 2 \\
    \left[
        \begin{array}{l}
            x=1 \\
            x=6
        \end{array}
    \right. 
    
\end{cases}
$ \\
$ \Leftrightarrow 
x=1 $ \\
Kết luận: Vậy phương trình có nghiệm duy nhất: $ x=1 $.

      	%%%%%%%%%%%%%%%%%%%%%%%%%%%%%%%%%%%%%%%%%%%%%%%%%%%%%%%%%%%%%%%%%%%%%%%%%%%%%%%%
%			                 vi du				       %
%%%%%%%%%%%%%%%%%%%%%%%%%%%%%%%%%%%%%%%%%%%%%%%%%%%%%%%%%%%%%%%%%%%%%%%%%%%%%%%%
\begin{vd}
  Giải phương trình: $ 2\left(\sqrt{2(2+x)}+2\sqrt{2-x}\right) = \sqrt{9x^2+16} $
\end{vd}
\begin{center}
    Giải:
\end{center}

Điều kiện: $ -2 \leq x \leq 2 $ \\
Với điều kiện trên phương trình tương đương : \\
$ 8(2+x) + 16\sqrt{2(4-x^2)} + 16(2-x) = 9x^2+16 $ \\
$ \Leftrightarrow 
asdfasfS    
$

      	%%%%%%%%%%%%%%%%%%%%%%%%%%%%%%%%%%%%%%%%%%%%%%%%%%%%%%%%%%%%%%%%%%%%%%%%%%%%%%%%
%			                 vi du				       %
%%%%%%%%%%%%%%%%%%%%%%%%%%%%%%%%%%%%%%%%%%%%%%%%%%%%%%%%%%%%%%%%%%%%%%%%%%%%%%%%
\begin{vd}
  Giải phương trình: $ 2\sqrt{x+2+2\sqrt{x+1}}-\sqrt{x+1}=4 $
\end{vd}
\begin{center}
    Giải:
\end{center}

Điều kiện: $ x\geq -1 $ \\
Với điều kiện trên phương trình tương đương : \\
$ 2\sqrt{(1 +\sqrt{x+1})^2} - \sqrt{x+1} =4 $ \\
$ \Leftrightarrow 
    2(1+\sqrt{x+1}) -\sqrt{x+1} =4 $\\
$ \Leftrightarrow 
    \sqrt{x+1} =2 $\\
$ \Leftrightarrow 
    x+1 =4 \Leftrightarrow 
    x=3$\\
Kết luận: Vậy phương trình có nghiệm duy nhất: $ x=3 $.

      	%%%%%%%%%%%%%%%%%%%%%%%%%%%%%%%%%%%%%%%%%%%%%%%%%%%%%%%%%%%%%%%%%%%%%%%%%%%%%%%%
%			                 vi du				       %
%%%%%%%%%%%%%%%%%%%%%%%%%%%%%%%%%%%%%%%%%%%%%%%%%%%%%%%%%%%%%%%%%%%%%%%%%%%%%%%%
\begin{vd}
  Giải phương trình: $ \sqrt{x-1+2\sqrt{x-2}} -\sqrt{x-1-2\sqrt{x-2}}=1 $
\end{vd}
\begin{center}
    Giải:
\end{center}

Điều kiện: $ x\geq 2 $ \\
Đặt $ t=\sqrt{x-2}, t\geq 0 \Rightarrow t^2 = x -2 \Leftrightarrow x =t^2 +2 $ \\
Khi đó phương trình tương đương: \\
$ \sqrt{t^2 + 1 +2t} - \sqrt{t^2 +1 -2t } =1 $ \\
$ \Leftrightarrow 
    \sqrt{(t+1)^2} - \sqrt{(t-1)^2} = 1 $ \\
$ \Leftrightarrow 
    t+1 - | t-1 | = 1 $ \\
$ \Leftrightarrow 
    \left[
        \begin{array}{lr}
            t+1 - (t -1) =1 & \left( t \geq 1\right) \\
            t +1 - (1-t) =1 & \left( t < 1 \right) 
        \end{array}
    \right.
$ \\
$ \Leftrightarrow 
    \left[ 
        \begin{array}{lrl}
            t+1-t+1=1 & \left( t\geq 1 \right) & \text{ (vn) }\\
            t+1 -1 +t =1 & \left( t<1 \right) &
        \end{array}
        \right.
$\\
$ \Leftrightarrow 
    t = \frac{1}{2} $ \\
$ \Leftrightarrow 
    \sqrt{x-2} = \frac{1}{2}
$\\
$ \Leftrightarrow x-2 = \frac{1}{4} $\\
$ \Leftrightarrow  x = \frac{9}{4}$\\
Kết luận: Vậy phương trình có nghiệm duy nhất: $ x=\frac{9}{4} $.

      	%%%%%%%%%%%%%%%%%%%%%%%%%%%%%%%%%%%%%%%%%%%%%%%%%%%%%%%%%%%%%%%%%%%%%%%%%%%%%%%%
%			                 vi du				       %
%%%%%%%%%%%%%%%%%%%%%%%%%%%%%%%%%%%%%%%%%%%%%%%%%%%%%%%%%%%%%%%%%%%%%%%%%%%%%%%%
\begin{vd}
Giải phương trình: $ \dfrac{x }{2} -2 = \dfrac{x^2}{2\left( 1+\sqrt{1+x }\right)^2} $
\end{vd}
\begin{center}
    Giải:
\end{center}

Điều kiện: $ x \geq -1  $ \\
Vì $x = 0$ không là nghiệm phương trình nên phương trình tương đương : \\
$ \dfrac{x }{2} -2 = \dfrac{x^2 \left( 1 -\sqrt{1+x }\right)^2}{2x^2} $\\
$ \Leftrightarrow  x -4 = 1 -2\sqrt{1+x } + 1+x $ \\
$ \Leftrightarrow        \sqrt{ 1+ x }  =3 $\\
$ \Leftrightarrow 1+x =9 $\\
$ \Leftrightarrow x = 8$\\
 So với điều kiện ta được nghiệm của phương trình: $ x=8 $.


      	%%%%%%%%%%%%%%%%%%%%%%%%%%%%%%%%%%%%%%%%%%%%%%%%%%%%%%%%%%%%%%%%%%%%%%%%%%%%%%%%
%			                 vi du				       %
%%%%%%%%%%%%%%%%%%%%%%%%%%%%%%%%%%%%%%%%%%%%%%%%%%%%%%%%%%%%%%%%%%%%%%%%%%%%%%%%
\begin{vd}
  Giải phương trình: $ 3\left( 2+\sqrt{x-2} \right) =2x +\sqrt{x+6} $
\end{vd}
\begin{center}
    Giải:
\end{center}

Điều kiện: $ x \geq 0 $ \\
Với điều kiện trên phương trình tương đương : \\
$ 3 \sqrt{x-2} -\sqrt{x+6} = 2(x-3) $\\
$ \Leftrightarrow 9 (x-2) - (x+6) = 2(x-3) \left( 3 \sqrt{x-2 } + \sqrt{x+6 } \right) $ \\
$ \Leftrightarrow 8(x - 3) =2(x-3) \left( 3\sqrt{x-2 } + \sqrt{x+6 } \right) $ \\
$ \Leftrightarrow (x-3) ( 4 - \left( 3\sqrt{x-2 } + \sqrt{x+6} \right) $ \\
$ \left[
    \begin{array}{lr}
        x  = 3 & \\
        3\sqrt{x-2 } + \sqrt{ x+ 6} -4 =0 & (1)  
    \end{array}
\right. $\\
Ta có $(1) \Leftrightarrow 3\sqrt{x-2 } + \sqrt{x+6} =4 $\\
$ \Leftrightarrow 9(x-2) + x+6 + 6\sqrt{(x-2)(x+6)} =16 $\\
$ \Leftrightarrow 10x - 12 +6 \sqrt{x^2 +4x -12 } =16 $ \\
$ \Leftrightarrow 3 \sqrt{x^2 +4x -12 } = 14 -5x $\\
$ \Leftrightarrow \begin{cases}
    x \leq  \dfrac{14}{5} \\
    9(x^2 + 4x -12) = (14 -5x)^2 
\end{cases} $ \\
$ \Leftrightarrow \begin{cases}
    x \leq \dfrac{14}{5} \\
    16x^2 - 176x +304 = 0
\end{cases} $ \\
$ \Leftrightarrow \begin{cases}
     x \leq \dfrac{14}{5} \\
     x^2 -11x + 19 =0
\end{cases} $ \\

$ \Leftrightarrow \begin{cases}
     x \leq \dfrac{14}{5} \\
     \left[
         \begin{array}{l}
             x = \dfrac{11+3 \sqrt{5}}{2} \\
             x = \dfrac{11-3 \sqrt{5}}{2} 
         \end{array}
     \right. 
     
\end{cases} $ \\ 
$ \Leftrightarrow 
             x = \dfrac{11-3 \sqrt{5}}{2}  $ \\ 

Kết luận: Vậy phương trình có 2 nghiệm là: $ x=3 ;  x = \dfrac{11-3 \sqrt{5}}{2}   $.


      	%%%%%%%%%%%%%%%%%%%%%%%%%%%%%%%%%%%%%%%%%%%%%%%%%%%%%%%%%%%%%%%%%%%%%%%%%%%%%%%%
%			                 vi du				       %
%%%%%%%%%%%%%%%%%%%%%%%%%%%%%%%%%%%%%%%%%%%%%%%%%%%%%%%%%%%%%%%%%%%%%%%%%%%%%%%%
\begin{vd}
    Giải phương trình: $ \sqrt[3]{x-1}+\sqrt[3]{x-2} = \sqrt[3]{2x-3}$
\end{vd}
\begin{center}
    Giải:
\end{center}

Phương trình đã cho tương đương: \\
$ \left( \sqrt[3]{x-1}+\sqrt[3]{x-2} \right)^3 = \left( \sqrt[3]{2x-3 } \right)^3 $\\ 
$ \Leftrightarrow x -1 + x -2 + 3 \sqrt[3]{x-1}\sqrt[3]{x-2}\left(  \sqrt[3]{x-1}+\sqrt[3]{x-2}\right) = 2x -3 $ \\
$ \Rightarrow   \sqrt[3]{x-1}\sqrt[3]{x-2}  \sqrt[3]{2x-3} =0 $ \\
$ \Leftrightarrow \left[
    \begin{array}{l}
        x-1 =0 \\
        x-2 =0 \\
        2x -3 =0
    \end{array}
\right. $ \\
$ \Leftrightarrow \left[
    \begin{array}{l}
        x=1 \\ x=2 \\ x= \dfrac{3}{2}
    \end{array}
\right. $ \\
Thử lại ta thấy nghiệm của phương trình là: $ x=1; x=2 ; x=\dfrac{3}{2}$.



      	%%%%%%%%%%%%%%%%%%%%%%%%%%%%%%%%%%%%%%%%%%%%%%%%%%%%%%%%%%%%%%%%%%%%%%%%%%%%%%%%
%			                 vi du				       %
%%%%%%%%%%%%%%%%%%%%%%%%%%%%%%%%%%%%%%%%%%%%%%%%%%%%%%%%%%%%%%%%%%%%%%%%%%%%%%%%
\begin{vd}
  Giải phương trình: $ \sqrt{2x^2 +x +6} + \sqrt{x^2 +x +2} = x + \frac{4}{x } $
\end{vd}
\begin{center}
    Giải:
\end{center}
$\sqrt{2x^2 +x +6} + \sqrt{x^2 +x +2} = x + \frac{4}{x } \text{ (1) }$
Điều kiện: $ x \neq 0  $ \\
Để $x$ là nghiệm của phương trình thì $x >0$. \\
Phương trinh đã cho tương đương:\\
$     \dfrac{x^2 +4}{\sqrt{2x^2 +x +6} - \sqrt{x^2 +x +2}} =     \dfrac{x^2 +4}{x}  $ \\
$ \Leftrightarrow \sqrt{2x^2 +x +6} - \sqrt{x^2 +x +2} = x \text{ (2) }$ \\
Kết hợp giữa (1) và (2) ta được phương trình: $ 2     \sqrt{x^2+x +2 } =     \dfrac{4}{x } $ \\
$ \Leftrightarrow x \sqrt{x^2 +x +2 } = 2 $ \\
$  \Leftrightarrow x^4 + x^3 + 2x^2 -4 =0 $ \\
$ \Leftrightarrow (x-1)( x^3 +2x^2 +4x +4) =0 $ \\
$ \Leftrightarrow x = 1 $ ( do $ x^3 +2x^2 +4x +4 > 0$  $  \forall x >0 $ ) \\
Kết luận: Vậy phương trình có nghiệm duy nhất: $ x=1 $.

 \section{Phương trình bậc hai.}



\section{Phương trình bậc ba .}

\section{Phương trình bậc bốn .}
% ============ HẾT CHƯƠNG I ==================
\Closesolutionfile{loigiaichung}
\section*{Lời giải bài tập chương 1}
\addcontentsline{toc}{section}{Lời giải bài tập chương 1}
\markright{Lời giải bài tập chương 1}
{\small\begin{Answer}{1}
This is an answer.
\end{Answer}
}
