\part{PHƯƠNG TRÌNH VÔ TỈ}%
\minitoc %

\thispagestyle{empty}
\Opensolutionfile{loigiaichung}[baitapC1]
\vspace*{1cm}
\chapter{Phương pháp giải phương trình vô tỉ}
\section{Phương trình vô tỉ giải được bằng phương pháp tương đương .}
  \subsection{Phương pháp giải }
    Chuyển vế đổi dấu để hai vế không âm, sau đó bình phương hai vế (ta được phương trình tương đương) để khử căn thức, đưa về phương trình đại số, trong đó: 
        \begin{itemize}
          \item Phương trình có dạng $	\sqrt{A}=B \Leftrightarrow \begin{cases} B \geq 0 \\ A = B^2 \end{cases}$
          \item Ta có thể bình phương mà không cần quan tâm tới điều kiện hai vế phải tương đương (ta được phương trình hệ quả) để khử căn thức, tuy nhiên sau khi giải ra nghiệm ta phải thử lại nghiệm.

        
      
        \end{itemize} 

      \subsection{Ví dụ minh họa}
\section{Phương trình bậc hai  .}



\section{Phương trình bậc ba .}

\section{Phương trình bậc bốn .}
% ============ HẾT CHƯƠNG I ==================
\Closesolutionfile{loigiaichung}
\section*{Lời giải bài tập chương 1}
\addcontentsline{toc}{section}{Lời giải bài tập chương 1}
\markright{Lời giải bài tập chương 1}
{\small\begin{Answer}{1}
This is an answer.
\end{Answer}
}
